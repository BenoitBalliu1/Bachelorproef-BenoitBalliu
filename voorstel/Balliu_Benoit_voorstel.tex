%==============================================================================
% Sjabloon onderzoeksvoorstel bachelorproef
%==============================================================================
% Gebaseerd op LaTeX-sjabloon ‘Stylish Article’ (zie voorstel.cls)
% Auteur: Jens Buysse, Bert Van Vreckem

\documentclass[fleqn,10pt]{voorstel}
\usepackage[T1]{fontenc}
\usepackage[utf8]{inputenc}
\usepackage{graphicx}
%------------------------------------------------------------------------------
% Metadata over het voorstel
%------------------------------------------------------------------------------

\JournalInfo{HoGent Bedrijf en Organisatie}
\Archive{Bachelorproef 2018 - 2019} % Of: Onderzoekstechnieken

%---------- Titel & auteur ----------------------------------------------------

% TODO: geef werktitel van je eigen voorstel op
\PaperTitle{Analyse, architectuur en proof-of-concept van een beveiligde omgeving voor het afnemen van examens per computer op eigen laptop.}
\PaperType{Onderzoeksvoorstel Bachelorproef} % Type document

% TODO: vul je eigen naam in als auteur, geef ook je emailadres mee!
\Authors{Benoit Balliu\textsuperscript{1}} % Authors
\CoPromotor{Bert Van Vreckem\textsuperscript{2} (Hogeschool Gent)}
\affiliation{\textbf{Contact:}
  \textsuperscript{1} \href{mailto:benoit.balliu.y9010@student.hogent.be}{benoit.balliu.y9010@student.hogent.be};
  \textsuperscript{2} \href{mailto:bert.vanvreckem@hogent.be}{bert.vanvreckem@hogent.be};
}

%---------- Abstract ----------------------------------------------------------

\Abstract{Om de hoge kosten van examens per computer die in computerlokalen afgenomen worden, op de Hogeschool Gent te verminderen, onderzoeken we in deze bachelorproef of er een goedkoper en minstens even veilig alternatief opgezet kan worden op laptops van de studenten zelf.Tijdens dit onderzoek wordt er enkel rekening gehouden met de examens van de Toegepaste Informatica. Aan het einde van dit onderzoek wordt een compleet functionele proof-of-concept verwacht. }
    
   % Hier schrijf je de samenvatting van je voorstel, als een doorlopende tekst van één paragraaf. Wat hier zeker in moet vermeld worden: \textbf{Context} (Waarom is dit werk belangrijk?); \textbf{Nood} (Waarom moet dit onderzocht worden?); \textbf{Taak} (Wat ga je (ongeveer) doen?); \textbf{Object} (Wat staat in dit document geschreven?); \textbf{Resultaat} (Wat verwacht je van je onderzoek?); \textbf{Conclusie} (Wat verwacht je van van de conclusies?); \textbf{Perspectief} (Wat zegt de toekomst voor dit werk?).
    
    %Bij de sleutelwoorden geef je het onderzoeksdomein, samen met andere sleutelwoorden die je werk beschrijven.
    
    %Vergeet ook niet je co-promotor op te geven.




\Keywords{Systeem- en netwerkbeheer --- Fraudebestrijding --- Examens per Computer} % Keywords
\newcommand{\keywordname}{Sleutelwoorden} % Defines the keywords heading name

%---------- Titel, inhoud -----------------------------------------------------

\begin{document}

\flushbottom % Makes all text pages the same height
\maketitle % Print the title and abstract box
\tableofcontents % Print the contents section
\thispagestyle{empty} % Removes page numbering from the first page

%------------------------------------------------------------------------------
% Hoofdtekst
%------------------------------------------------------------------------------

% De hoofdtekst van het voorstel zit in een apart bestand, zodat het makkelijk
% kan opgenomen worden in de bijlagen van de bachelorproef zelf.
%---------- Inleiding ---------------------------------------------------------

\section{Introductie} % The \section*{} command stops section numbering
\label{sec:introductie}

 Momenteel worden de meeste computerexamens op Hogeschool Gent in een beveiligde omgeving op een computer van de hogeschool afgenomen. Deze methodiek zorgt voor een hoge kost. De hogeschool moet een groot aantal computers te beschikking stellen. Deze moeten beschikken over software eigen aan het examen, en monitoringsoftware bevatten zodat examens in een beveiligde omgeving afgelegd kunnen worden. \\ In deze bachelorproef onderzoeken we de haalbaarheid van een beveiligde omgeving waar studenten computerexamens op hun eigen laptop kunnen afleggen. Deze beveilige omgeving moet maximaal vermijden dat fraude mogelijk is. \\

Deelonderzoeksvragen:
  \begin{itemize}
     \item Hoe gebeuren computerexamens nu op de hogeschool? Welke tools worden gebruikt? Welke regels en beperkingen leggen lectoren nu typisch op bij computerexamens?
     
     \item Welke tools/oplossingen bestaan er tegenwoordig voor dit soort situaties? Zijn die voldoende flexibel om bijvoorbeeld een examen programmeren of andere ict-vakken te faciliteren?
     
     \item Als we zelf een omgeving willen opzetten, welke componenten moet die dan bevatten? Welke beperkingen kunnen we studenten opleggen en welk gedrag kunnen we niet vermijden?
  \end{itemize}

%Hier introduceer je werk. Je hoeft hier nog niet te technisch te gaan.
%Je beschrijft zeker:
%\begin{itemize}
%  \item de probleemstelling en context
%  \item de motivatie en relevantie voor het onderzoek
%  \item de doelstelling en onderzoeksvraag/-vragen
%\end{itemize}

%---------- Stand van zaken ---------------------------------------------------

\section{Literatuurstudie}
\label{sec:literatuurstudie}

\subsection{Voordelen van bring-your-own-device examens}

Studenten die een examen afleggen op hun eigen hardware zijn vertrouwd met hun toestel. Waardoor hun stressgevoeligheid afneemt, volgens
 \textcite{TeckSwee2014}. In dat onderzoek was er op een populatie van 672 studenten 74.02\% blij met deze manier om examens af te nemen. 

\subsection{Problemen bij bring-your-own-device examens}
\subsubsection{Fraude}
Aangezien examens afleggen op eigen hardware een relatief nieuw gegeven is, ervaren we nog enkele kinderziektes.\\ \textcite{Dawson2016} bekeek in zijn onderzoek 5 manieren om fraude te plegen tijdens een examen op eigen laptop, namelijk: \\
\begin{itemize}
    \item De examenopgave lokaal opslaan en achteraf online plaatsen.
    \item Het examen op een virtuele omgeving afleggen en op die manier het besturingssysteem manipuleren
    \item Scripts uitvoeren via usb-keyboard hacks
    \item Software aanpassingen maken 
    \item Cold-boot aanval\\
\end{itemize}


Volgens het onderzoek van \textcite{VegendiaSindre2015} is het vermijden van elektronische communicatie een prioriteit. Dit is dan ook iets wat in de proof-of-concept onmogelijk gemaakt moet worden. 

\subsubsection{Overige problemen}

Naast fraude zijn er andere aandachtspunten bij het afnemen van examens op eigen hardware, \textcite{Hillier2015} heeft het in zijn onderzoek over onder andere: laptops die niet sterk genoeg zijn om bepaalde software aan te kunnen, onverwachte crashes van software of besturingssystemen, hardware die tijdens het examen faalt en batterijcapaciteit (indien er geen toegang tot stroom is). 



% Voor literatuurverwijzingen zijn er twee belangrijke commando's:
% \autocite{KEY} => (Auteur, jaartal) Gebruik dit als de naam van de auteur
%   geen onderdeel is van de zin.
% \textcite{KEY} => Auteur (jaartal)  Gebruik dit als de auteursnaam wel een
%   functie heeft in de zin (bv. ``Uit onderzoek door Doll & Hill (1954) bleek
%   ...'')


%---------- Methodologie ------------------------------------------------------
\section{Methodologie}
\label{sec:methodologie}

Voor dit onderzoek maken we een Analyse en beslissen we over de scope en de vereisten van de beveiligde omgeving. We gaan een onderzoek doen naar huidige oplossingen voor bring-your-own-device examens. \\ Via een rondvraag bij docenten gaan we onderzoeken welke vereisten elk computerexamen heeft. Zo kunnen we in de proof-of-concept een preset maken voor elk specifiek examen op computer. \\ De proof-of-concept zelf zal een volledig virtuele omgeving zijn, daarvoor gaan we o.a. gebruik maken van GNS3 en VirtualBox. 

%---------- Verwachte resultaten ----------------------------------------------
\section{Verwachte resultaten}
\label{sec:verwachte_resultaten}

Aan het einde van dit onderzoek verwachten we een compleet werkende en veilige proof-of-concept omgeving te hebben. De bedoeling is dat docenten hiermee vlot hun examens kunnen voorbereiden en verzekerd zijn van de fraudebestendigheid.


%---------- Verwachte conclusies ----------------------------------------------
\section{Verwachte conclusies}
\label{sec:verwachte_conclusies}

We verwachten dat een dergelijke omgeving op te zetten is. Al vrezen we wel dat er enkele beperkingen aan dit systeem zullen zijn. Bijvoorbeeld: aangezien het niet de bedoeling is dat er software op de laptop van een student geïnstalleerd wordt, zal je extra informatie die een student op zijn laptop zelf staan heeft niet kunnen verbieden. 



%------------------------------------------------------------------------------
% Referentielijst
%------------------------------------------------------------------------------
% TODO: de gerefereerde werken moeten in BibTeX-bestand ``voorstel.bib''
% voorkomen. Gebruik JabRef om je bibliografie bij te houden en vergeet niet
% om compatibiliteit met Biber/BibLaTeX aan te zetten (File > Switch to
% BibLaTeX mode)

\phantomsection
\printbibliography

\end{document}
