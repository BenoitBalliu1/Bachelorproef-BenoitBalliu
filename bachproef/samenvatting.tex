%%=============================================================================
%% Samenvatting
%%=============================================================================

% TODO: De "abstract" of samenvatting is een kernachtige (~ 1 blz. voor een
% thesis) synthese van het document.
%
% Deze aspecten moeten zeker aan bod komen:
% - Context: waarom is dit werk belangrijk?
% - Nood: waarom moest dit onderzocht worden?
% - Taak: wat heb je precies gedaan?
% - Object: wat staat in dit document geschreven?
% - Resultaat: wat was het resultaat?
% - Conclusie: wat is/zijn de belangrijkste conclusie(s)?
% - Perspectief: blijven er nog vragen open die in de toekomst nog kunnen
%    onderzocht worden? Wat is een mogelijk vervolg voor jouw onderzoek?
%
% LET OP! Een samenvatting is GEEN voorwoord!

%%---------- Nederlandse samenvatting -----------------------------------------
%
% TODO: Als je je bachelorproef in het Engels schrijft, moet je eerst een
% Nederlandse samenvatting invoegen. Haal daarvoor onderstaande code uit
% commentaar.
% Wie zijn bachelorproef in het Nederlands schrijft, kan dit negeren, de inhoud
% wordt niet in het document ingevoegd.


%%---------- Samenvatting -----------------------------------------------------
% De samenvatting in de hoofdtaal van het document

\chapter*{Samenvatting}

Deze bachelorproef is gemaakt in opdracht van Hogeschool Gent, zij willen graag hun huidig systeem om examens op computer af te nemen moderniseren. Momenteel beheert de afdeling Systeembeheer van Hogeschool Gent een groot aantal desktops en softwarepakketten in combinatie met NetSupport School, maar daar loopt af en toe iets mis. Het komt wel vaker voor dat examens niet automatisch opgehaald kunnen worden of examens niet door kunnen gaan omdat de juiste software niet ge\"{i}nstalleerd is. 

Voor het nieuwe systeem wil Hogeschool Gent de mogelijkheid van bring-your-own-device examens onderzoeken. Dit is een systeem waarbij de studenten de examens afleggen op hun eigen laptop. De grootste uitdaging van dergelijk systeem is deze laptops genoeg beveiligen tegen fraude. Voor dit onderzoek werd uitdrukkelijk gevraagd om te onderzoeken of zo een systeem haalbaar is zonder dat studenten hiervoor speciale software op hun systeem dienen te installeren.

Na het onderzoeken van enkele systemen is er besloten een proof-of-concept op te stellen van een beveiligde netwerkomgeving. Hoewel deze omgeving alle vormen van fraude via het internet voorkomt, kan er niet uitgesloten worden dat een student opgeloste oefeningen of een cursus op zijn laptop heeft staan. Dit systeem voldoet dus niet aan de eisen van de Hogeschool Gent want het zou een stuk minder veilig zijn dan de methode die ze vandaag hanteert. 

De belangrijkste conclusie van dit onderzoek is dat fraude op laptops van de studenten niet kan uitgesloten worden zonder dat er speciale software (bijvoorbeeld monitoring-software) ge\"{i}nstalleerd wordt. Het lijkt mij daarom interessant om het onderzoek naar bring-your-own-device examens opnieuw te voeren met andere uitgangspunten. Mogelijks lijdt een aangepast onderzoek tot nieuwe inzichten.
