%%=============================================================================
%% Conclusie
%%=============================================================================

\chapter{Conclusie}
\label{ch:conclusie}

%% TODO: Trek een duidelijke conclusie, in de vorm van een antwoord op de
%% onderzoeksvra(a)g(en). Wat was jouw bijdrage aan het onderzoeksdomein en
%% hoe biedt dit meerwaarde aan het vakgebied/doelgroep? Reflecteer kritisch
%% over het resultaat. Had je deze uitkomst verwacht? Zijn er zaken die nog
%% niet duidelijk zijn? Heeft het onderzoek geleid tot nieuwe vragen die
%% uitnodigen tot verder onderzoek?


In dit onderzoek is gezocht naar een antwoord op de vraag: 'Is het mogelijk een omgeving, die beveiligd is en voldoet aan de eisen van Hogeschool Gent, op te stellen waar studenten computerexamens op hun eigen laptop kunnen afleggen?'

Uit het onderzoek naar een nieuwe, BYOD, examenomgeving is gebleken dat dergelijke omgeving nooit genoeg beveiligd kan worden tegen fraude. Dit natuurlijk binnen de scope die aan het onderzoek gegeven is, namelijk, het is niet de bedoeling dat er software op apparaten van de studenten ge\"{\i}nstalleerd wordt. De beveiligde netwerkomgeving, zoals voorgesteld in het hoofdstuk proof-of-concept, is de enige omgeving die tegen fraude via het internet beschermt. In dergelijke omgeving is er geen controle op de documenten waarover de student beschikt. De nieuwe examenomgeving zou minstens het zelfde niveau van beveiliging als het huidige NetSupport School systeem moeten halen. Binnen de opgelegde scope is dit niet mogelijk. 

Het principe van bring-your-own-device examens is wel interessant gebleken, vooral door de lagere kosten en de kleinere administratieve overhead. Het zou daarom opportuun zijn om dit onderzoek te hernemen, met een gewijzigde scope. Indien bij het onderzoeken van een BYOD examenomgeving, het installeren van (kiosk)software op apparaten van studenten toegestaan is, kan dergelijk onderzoek mogelijks een ander resultaat opleveren.




