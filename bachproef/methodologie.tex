%%=============================================================================
%% Methodologie
%%=============================================================================

\chapter{Methodologie}
\label{ch:methodologie}

%% TODO: Hoe ben je te werk gegaan? Verdeel je onderzoek in grote fasen, en
%% licht in elke fase toe welke stappen je gevolgd hebt. Verantwoord waarom je
%% op deze manier te werk gegaan bent. Je moet kunnen aantonen dat je de best
%% mogelijke manier toegepast hebt om een antwoord te vinden op de
%% onderzoeksvraag.

\section{Fasering}

De rest van dit onderzoek is onderverdeeld in 3 stukken:
\begin{itemize}
	\item Onderzoek naar mogelijke oplossingen
	\item Mogelijke opstellingen maken
	\item Opstellen van een proof of concept
\end{itemize}

\section{Onderzoek naar mogelijke oplossingen}
Nu er beslist is welk systeem er opgezet wordt voor de proof-of-concept is het belangrijk op te lijsten hoe het onderzoek verder verloopt. Hieronder vindt u een oplijsting van technologie\"{e}n die gebruikt kunnen worden bij de de eindelijke opstelling van het bring-your-own-device systeem met beveiligde netwerkomgeving.

\subsection{GNS3}
GNS3 ofwel Graphical Network Simulator 3 is een network software emulator. Deze software staat toe om virtuele en fysieke devices met elkaar te verbinden om zo complexe netwerken te simuleren. GNS3 kan virtuele machines van VMWare, VirtualBox of KVM gewoon importeren en routeren. Dankzij GNS3 was het testen van een opstelling tijdens het onderzoek een stuk makkelijker en kon het opstellen van een proof of concept versneld worden.

\subsubsection{Tekortkoming}
Wireless access points zijn niet ondersteund binnen GNS3. Voor de proof of concept is er dus gebruik gemaakt van een opstelling met wired connections. In de realiteit verkiezen we dus wireless access omdat niet alle moderne laptops nog een ethernetpport hebben.   

\subsection{Virtualisatie van overige apparaten}
Om de virtualisatie van niet-netwerk apparaten, zoals end devices te faciliteren is er gebruik gemaakt van VMWare. Gebruik van Virtualbox machines of machines op een KVM Server is ook mogelijk. 


\subsection{Open-source software}
Omdat de opstelling open-source moest blijven is er voor de opstelling enkel naar open source netwerk operating systems/software gekeken. Onder andere:
\begin{itemize}	
	\item pfSense: Open-source operating system gebaseerd op FreeBSD. Router/firewall.
	\item OpenWrt: Open-source Linux operating system. Volledige netwerk oplossing. 
	\item dnsmasq: Open-source Unix-like operating system. DNS forwarder en DHCP Server.
\end{itemize}


\subsection{Automatisatie}

Om het opstellen van dergelijke omgeving zo simpel mogelijk te maken is er gekozen om een groot stuk te automatiseren d.m.v. Ansible. Ansible is Agentless SSH automation. Dit wil zeggen dat er op de nodes die je beheert met ansible geen speciale software moeten draaien, enkel een OpenSSH server, deze is standaard geinstalleerd om de meeste systemen. Met Ansible en haar netwerk modules kan je makkelijk volledige netwerkomgeivngen opzetten zonder zelf aan de command-line interface te moeten komen. 

\section{Mogelijke opstellingen maken}


\section{Proof of concept}
Na het configureren van enkele opstellingen is er \'{e}\'{e}n uitgekozen. Degene die het makkelijkst was om op te zetten en het meest simpel om te  configureren. Daarover kan u in het volgende hoofdstuk lezen.
