%%=============================================================================
%% Methodologie
%%=============================================================================

\chapter{Methodologie}
\label{ch:methodologie}

%% TODO: Hoe ben je te werk gegaan? Verdeel je onderzoek in grote fasen, en
%% licht in elke fase toe welke stappen je gevolgd hebt. Verantwoord waarom je
%% op deze manier te werk gegaan bent. Je moet kunnen aantonen dat je de best
%% mogelijke manier toegepast hebt om een antwoord te vinden op de
%% onderzoeksvraag.

\section{Fasering}

De rest van dit onderzoek is uitgevoerd in 3 delen, namelijk: 
\begin{itemize}
	\item Eerst is er onderzocht welke oplossingen er reeds bestaan voor dergelijke systemen, en hoe dit best opgesteld wordt in de proof-of-concept.
	\item Dan zijn er een aantal mogelijke opstellingen gemaakt en met elkaar vergeleken.
	\item Uiteindelijk is er een proof-of-concept opgesteld, rekeninghoudend met de bevindingen uit de tweede fase.
\end{itemize}

\section{Onderzoek naar mogelijke oplossingen}
Nu er beslist is welk systeem er opgezet wordt voor de proof-of-concept is het belangrijk op te lijsten hoe het onderzoek verder verloopt. Hieronder vindt u een oplijsting van technologie\"{e}n die gebruikt kunnen worden bij de uiteindelijke opstelling van het bring-your-own-device systeem met beveiligde netwerkomgeving.

\subsection{GNS3}
GNS3 ofwel Graphical Network Simulator 3 is een network software emulator. Deze software staat toe om virtuele en fysieke apparaten met elkaar te verbinden om zo complexe netwerken te simuleren. GNS3 kan virtuele machines van VMWare, VirtualBox of KVM gewoon importeren en routeren. Dankzij GNS3 was het testen van een opstelling tijdens het onderzoek een stuk makkelijker en kon het opstellen van een proof-of-concept versneld worden.

\subsubsection{Tekortkoming}
Wireless access points zijn niet ondersteund binnen GNS3. Voor de proof-of-concept is er dus gebruik gemaakt van een opstelling met wired connections. In de realiteit verkiezen we dus wireless access omdat niet alle moderne laptops nog een ethernetpoort hebben.   

\subsection{Virtualisatie van overige apparaten}
Om de virtualisatie van niet-netwerk apparaten, zoals end devices te faciliteren is er gebruik gemaakt van VMWare. Gebruik van Virtualbox machines of machines op een KVM Server is ook mogelijk. 


\subsection{Open-source software}
Omdat de opstelling open-source moest blijven is er voor de opstelling enkel naar open source netwerk operating systems/software gekeken. Onder andere:
\begin{itemize}	
	\item pfSense: Open-source operating system gebaseerd op FreeBSD. Router/firewall.
	\item OpenWrt: Open-source Linux operating system. Volledige netwerk oplossing. 
	\item dnsmasq: Open-source Unix-like operating system. DNS forwarder en DHCP Server.
\end{itemize}


\subsection{Automatisatie}

Om het opstellen van dergelijke omgeving zo simpel mogelijk te maken is er gekozen om een groot stuk te automatiseren d.m.v. Ansible. Ansible is Agentless SSH automation. Dit wil zeggen dat er op de nodes die je beheert met ansible geen speciale software moet draaien, enkel een OpenSSH server die standaard geinstalleerd is op de meeste systemen. Met Ansible en haar netwerk modules kan je makkelijk volledige netwerkomgevingen opzetten zonder zelf aan de command-line interface te moeten wijzigenn. 

\section{Mogelijke opstellingen maken}
Voor er een goede finale opstelling opgezet kan worden moeten er een aantal opstellingen met elkaar vergeleken worden. Hier volgt een lijst van opstellingen die suboptimaal bleken na onderzoek:
\begin{itemize}
	\item \textbf{Opstelling waarbij de netwerkinfrastructuur uit ciscoapparatuur bestaat}. Na het opzetten van die opstelling in GNS3 bleek deze suboptimaal. Dergelijk systeem is heel erg duur en voor het beperkt aantal gebruikers dat op hetzelfde moment met de apparatuur moet verbinden zou cisco-infrastructuur onnodig zijn. Cisco levert enterprise-grade infrastructuur, in dit geval is die apparatuur ongeschikt. 
	\item \textbf{Opstelling met OpenWRT router}. OpenWRT is gemaakt om op veel kleinere en minder krachtige apparaten te werken. Bij het opzetten van een OpenWRT omgeving was het duidelijk dat de configuratie van de router niet zo gebruiksvriendelijk is. Aangezien het de bedoeling is dat lectoren zelf de omgeving beheren is het gebruik van OpenWRT niet aangeraden. 
	\item \textbf{Opstelling met pfSense Router zonder Ansible}. Het gebruik van pfSense als router software maakt het opstellen van de netwerkomgeving een stuk makkelijker. Lectoren die omgeving volledig zelf laten opzetten is niet haalbaar.  
	 
\end{itemize} 

\section{Proof-of-concept}
Na het configureren van die vorige opstellingen is er \'{e}\'{e}n uitgekozen, namelijk degene die het best voldoet aan de vastgestelde vereisten. Daarover kan u in het volgende hoofdstuk lezen.
