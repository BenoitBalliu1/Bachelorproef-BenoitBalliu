%%=============================================================================
%% LaTeX sjabloon voor bachelorproef, HoGent Bedrijf en Organisatie
%% Opleiding Toegepaste Informatica
%%=============================================================================

\documentclass[fleqn,a4paper,12pt]{book}

\input{structure}

%%---------- Documenteigenschappen --------------------------------------------
%% TODO: Vul dit aan met je eigen info:

% Je eigen naam
\newcommand{\student}{Benoit Balliu}

% De naam van je promotor (lector van de opleiding)
\newcommand{\promotor}{Bert Van Vreckem}

% De naam van je co-promotor. Als je promotor ook je opdrachtgever is en je
% dus ook inhoudelijk begeleidt (en enkel dan!), mag je dit leeg laten.
\newcommand{\copromotor}{}

% Indien je bachelorproef in opdracht van/in samenwerking met een bedrijf of
% externe organisatie geschreven is, geef je hier de naam. Zoniet laat je dit
% zoals het is.
\newcommand{\instelling}{HoGent Faculteit Bedrijf en Organisatie}

% De titel van het rapport/bachelorproef
\newcommand{\titel}{Analyse, architectuur en proof-of-concept van een
beveiligde omgeving voor het afnemen van
computerexamens op eigen laptop}

% Datum van indienen (gebruik telkens de deadline, ook al geef je eerder af)
\newcommand{\datum}{31 mei 2019}

% Academiejaar
\newcommand{\academiejaar}{2018-2019}

% Examenperiode
%  - 1e semester = 1e examenperiode => 1
%  - 2e semester = 2e examenperiode => 2
%  - tweede zit  = 3e examenperiode => 3
\newcommand{\examenperiode}{1}

%%=============================================================================
%% Inhoud document
%%=============================================================================

\begin{document}

%---------- Taalselectie ------------------------------------------------------
% Als je je bachelorproef in het Engels schrijft, haal dan onderstaande regel
% uit commentaar. Let op: de tekst op de voorkaft blijft in het Nederlands, en
% dat is ook de bedoeling!

%\selectlanguage{english}

%---------- Titelblad ---------------------------------------------------------
\inserttitlepage

%---------- Samenvatting, voorwoord -------------------------------------------
\usechapterimagefalse
%%=============================================================================
%% Voorwoord
%%=============================================================================

\chapter*{Woord vooraf}
\label{ch:voorwoord}

%% TODO:
%% Het voorwoord is het enige deel van de bachelorproef waar je vanuit je
%% eigen standpunt (``ik-vorm'') mag schrijven. Je kan hier bv. motiveren
%% waarom jij het onderwerp wil bespreken.
%% Vergeet ook niet te bedanken wie je geholpen/gesteund/... heeft

Voor u ligt de bachelorproef: 'Analyse, architectuur en proof-of-concept van een beveiligde omgeving voor het afnemen van computerexamens op eigen laptop.' Het onderzoek hiervoor is in opdracht van Hogeschool Gent gevoerd. Deze bachelorproef is geschreven in het kader van mijn afstuderen aan de opleiding Toegepaste Informatica met afstudeerrichting Systeem- en Netwerkbeheer aan de Hogeschool Gent.

De deelonderzoeksvragen van dit onderzoek zijn opgesteld door mijn promotor en co-promotor Bert Van Vreckem. Bij deze wil ik hem graag bedanken voor de samenwerking en het extra inzicht binnen het examensysteem van Hogeschool Gent dat hij mij aanleverde. 

Verder wil ik lectoren Chris Arents (Hogeschool Gent, Aalst) en Heidi Roobrouck (Hogeschool Gent, Gent) bedanken voor hun tijd, dankzij hun heb ik de pijnpunten van het huidige systeem en de vereisten van het nieuwe systeem in kaart kunnen brengen.

Tot slot wil ik graag mijn ouders en mijn vriendin bedanken, zij hebben me tijdens dit onderzoek moreel ondersteund en hebben mijn bachelorproef nagelezen.

Ik wens u veel leesplezier toe.

Benoit Balliu \\
Gent, 24 mei 2019 
%%=============================================================================
%% Samenvatting
%%=============================================================================

% TODO: De "abstract" of samenvatting is een kernachtige (~ 1 blz. voor een
% thesis) synthese van het document.
%
% Deze aspecten moeten zeker aan bod komen:
% - Context: waarom is dit werk belangrijk?
% - Nood: waarom moest dit onderzocht worden?
% - Taak: wat heb je precies gedaan?
% - Object: wat staat in dit document geschreven?
% - Resultaat: wat was het resultaat?
% - Conclusie: wat is/zijn de belangrijkste conclusie(s)?
% - Perspectief: blijven er nog vragen open die in de toekomst nog kunnen
%    onderzocht worden? Wat is een mogelijk vervolg voor jouw onderzoek?
%
% LET OP! Een samenvatting is GEEN voorwoord!

%%---------- Nederlandse samenvatting -----------------------------------------
%
% TODO: Als je je bachelorproef in het Engels schrijft, moet je eerst een
% Nederlandse samenvatting invoegen. Haal daarvoor onderstaande code uit
% commentaar.
% Wie zijn bachelorproef in het Nederlands schrijft, kan dit negeren, de inhoud
% wordt niet in het document ingevoegd.


%%---------- Samenvatting -----------------------------------------------------
% De samenvatting in de hoofdtaal van het document

\chapter*{Samenvatting}

Deze bachelorproef is gemaakt in opdracht van Hogeschool Gent, zij willen graag hun huidig systeem om examens op computer af te nemen moderniseren. Momenteel beheert de afdeling Systeenbeheer van Hogeschool Gent een groot aantal desktops en softwarepakketten in combinatie met NetSupport School, maar daar loopt af en toe iets mis. Het komt wel vaker voor dat examens niet automatisch opgehaald kunnen worden of examens niet door kunnen gaan omdat de juiste software niet ge\"{i}nstalleerd is. 

Voor het nieuwe systeem wil Hogeschool Gent de mogelijkheid van bring-your-own-device examens onderzoeken. Dit is een systeem waarbij de studenten de examens afleggen op hun eigen laptop. De grootste uitdaging van dergelijk systeem is deze laptops genoeg beveiligen tegen fraude. Voor dit onderzoek werd uitdrukkelijk gevraagd om te onderzoeken of zo een systeem haalbaar is zonder dat studenten hiervoor speciale software op hun systeem dienen te installeren.

Na het onderzoeken van enkele systemen is er besloten een proof-of-concept op te stellen van een beveiligde netwerkomgeving. Hoewel deze omgeving alle vormen van fraude via het internet voorkomt, kan er niet uitgesloten worden dat een student opgeloste oefeningen of een cursus op zijn laptop heeft staan. Dit systeem voldoet dus niet aan de eisen van de Hogeschool Gent want het zou een stuk minder veilig zijn dan de methode die ze vandaag hanteert. 

De belangrijkste conclusie van dit onderzoek is dat fraude op laptops van de studenten niet kan uitgesloten worden zonder dat er speciale software (bijvoorbeeld monitoring-software) ge\"{i}nstalleerd wordt. Het lijkt mij daarom interessant om het onderzoek naar bring-your-own-device examens opnieuw te voeren met andere eisen. Het resultaat van dergelijk onderzoek kan een geheel ander resultaat opleveren wanneer het toegestaan is om software te installeren op de laptops van studenten. 


%---------- Inhoudstafel ------------------------------------------------------
\pagestyle{empty} % No headers
\tableofcontents % Print the table of contents itself
\cleardoublepage % Forces the first chapter to start on an odd page so it's on the right
\pagestyle{fancy} % Print headers again

%---------- Lijst figuren, afkortingen, ... -----------------------------------

% Indien gewenst kan je hier een lijst van figuren/tabellen opgeven. Geef in
% dat geval je figuren/tabellen altijd een korte beschrijving:
%
%  \caption[korte beschrijving]{uitgebreide beschrijving}

\listoffigures


% Als je een lijst van afkortingen of termen wil toevoegen, dan hoort die
% hier thuis. Gebruik bijvoorbeeld de ``glossaries'' package.
% https://www.sharelatex.com/learn/Glossaries

%%---------- Kern -------------------------------------------------------------

%%=============================================================================
%% Inleiding
%%=============================================================================

\chapter{Inleiding}
\label{ch:inleiding}



 Momenteel worden de meeste computerexamens op Hogeschool Gent in een beveiligde omgeving op een computer van de hogeschool afgenomen. Deze methodiek brengt enkele nadelen met zich mee. Het zorgt voor een hoge kost, de hogeschool moet een groot aantal computers te beschikking stellen en deze moeten beschikken over software eigen aan het examen, en monitoringsoftware bevatten zodat examens in een beveiligde omgeving afgelegd kunnen worden. \\ In deze bachelorproef onderzoeken we de haalbaarheid van een beveiligde omgeving waar studenten computerexamens op hun eigen laptop kunnen afleggen. Deze beveilige omgeving moet maximaal vermijden dat fraude mogelijk is. \\


\section{Probleemstelling}
\label{sec:probleemstelling}

Deze bachelorproef is in opdracht van Hogeschool Gent zelf. Het huidige systeem om computerexamens af te nemen op de Hogeschool is verouderd en te duur. Het zorgt voor een grote overhead voor de systeembeheerders en te veel werk voor de docenten zelf. Een systeem dat effici\"{e}nter en meer budgetvriendelijk is moet het huidige systeem vervangen.



\section{Onderzoeksvraag}
\label{sec:onderzoeksvraag}
Is het mogelijk een omgeving, die beveiligd is en voldoet aan de eisen van Hogeschool Gent, op te stellen waar studenten computerexamens op hun eigen laptop kunnen afleggen? 

\subsection{Deelonderzoeksvragen}
\begin{itemize}
	 \item Hoe gebeuren computerexamens nu op de hogeschool? Welke tools worden gebruikt? Welke regels en beperkingen leggen lectoren nu typisch op bij computerexamens?
	
	\item Welke tools/oplossingen bestaan er tegenwoordig voor dit soort situaties? Zijn die voldoende flexibel om bijvoorbeeld een examen programmeren of andere ict-vakken te faciliteren?
	
	\item Als we zelf een omgeving willen opzetten, welke componenten moet die dan bevatten? Welke beperkingen kunnen we studenten opleggen en welk gedrag kunnen we niet vermijden?

\end{itemize} 

\section{Onderzoeksdoelstelling}
\label{sec:onderzoeksdoelstelling}

Het doel van dit onderzoek is om een proof op concept op te stellen die voldoet aan de eisen van de systeembeheerders en de docenten en die een verbetering is tegenover het huidige systeem (op vlak van budget en gebruiksvriendelijkheid). 

\section{Opzet van deze bachelorproef}
\label{sec:opzet-bachelorproef}

% Het is gebruikelijk aan het einde van de inleiding een overzicht te
% geven van de opbouw van de rest van de tekst. Deze sectie bevat al een aanzet
% die je kan aanvullen/aanpassen in functie van je eigen tekst.

De rest van deze bachelorproef is als volgt opgebouwd:

In Hoofdstuk~\ref{ch:stand-van-zaken} wordt een overzicht gegeven van de stand van zaken binnen het onderzoeksdomein, op basis van een literatuurstudie.

In Hoofdstuk~\ref{ch:methodologie} wordt de methodologie toegelicht en worden de gebruikte onderzoekstechnieken besproken om een antwoord te kunnen formuleren op de onderzoeksvragen.

% TODO: Vul hier aan voor je eigen hoofstukken, één of twee zinnen per hoofdstuk
In hoofdstuk~\ref{ch:opstelling} wordt de finale opstelling van het byod-systeem voorgesteld.  

In Hoofdstuk~\ref{ch:conclusie}, tenslotte, wordt de conclusie gegeven en een antwoord geformuleerd op de onderzoeksvragen. Daarbij wordt ook een aanzet gegeven voor toekomstig onderzoek binnen dit domein.


\chapter{Stand van zaken}
\label{ch:stand-van-zaken}

% Tip: Begin elk hoofdstuk met een paragraaf inleiding die beschrijft hoe
% dit hoofdstuk past binnen het geheel van de bachelorproef. Geef in het
% bijzonder aan wat de link is met het vorige en volgende hoofdstuk.

% Pas na deze inleidende paragraaf komt de eerste sectiehoofding.

Voor er onderzoek gedaan wordt naar oplossing voor examens op eigen computer moet er gekeken worden naar de huidige situatie op de Hogeschool Gent en hoe andere instellingen deze probleemstelling aanpakken. 
%Dit hoofdstuk bevat je literatuurstudie. De inhoud gaat verder op de inleiding, maar zal het onderwerp van de bachelorproef *diepgaand* uitspitten. De bedoeling is dat de lezer na lezing van dit hoofdstuk helemaal op de hoogte is van de huidige stand van zaken (state-of-the-art) in het onderzoeksdomein. Iemand die niet vertrouwd is met het onderwerp, weet er nu voldoende om de rest van het verhaal te kunnen volgen, zonder dat die er nog andere informatie moet over opzoeken \autocite{Pollefliet2011}.

\section{Examens op Hogeschool Gent}

\subsection{Soorten examens}
Op Hogent, Faculteit Bedrijf en Organisatie worden er 2 soorten examens afgenomen, mondelinge en schriftelijke. Enkele schriftelijke examens worden enkel op papier afgelegd, anderen deels of volledig op computer. Dit zijn de soorten computerexamens die op HoGent afgenomen worden.
\begin{itemize}
\item Een online test waarbij antwoorden via de webbrowser invgevuld worden (bv. via Chamilo, Cisco-platform, enz.)
\item Een schriftelijk examen waaarbij de voorbereiding op pc gebeurt maar de antwoorden op papier ingevuld worden.
\item Een examen dat op computer m.b.v. specifieke software (bv. IDE en compiler) gemaakt wordt waarna het resultaat digitaal ingediend wordt (bv. Word document met antwoorden, zip-bestand met broncode, Github, ...)
\end{itemize}


\subsubsection{Schriftelijke examens (voorbereiding op computer) }

Deze examens worden altijd op desktops van HoGent afgenomen. In een beveiligde gemonitorde omgeving, waarin je enkel kan wat toegestaan is door de docent. De examenopzichter heeft via de admin-computer zicht op alle bureaubladen van de studenten die het examen aan het afleggen zijn. 

\paragraph{Voordelen}
Dit is een heel erg veilige manier om een gedeeltelijk computerexamen af te leggen. De student kan via die computer niets doen wat hij tijdens het examen niet mag doen. Aangezien het examen nog steeds op papier ingediend wordt is er ook geen extra werk voor de opzichter. Deze manier beschermt natuurlijk niet tegen een student die een spiekbriefje bij zich heeft of op zijn telefoon enkele dingen opzoekt, maar die controle moet door de examenopzichter uitgevoerd worden en in dit onderzoek wordt daarmee ook geen rekening gehouden.

\paragraph{Nadelen}
Deze manier kost handenvol geld. Per computer moeten er licenties betaald worden. Aangezien er een groot aantal examens tergelijkertijd afgelegd worden moet er ook een groot aantal computers over een dergelijke licenties beschikken.


\subsubsection{Schriftelijke examens (volledig op computer met behulp van software)}

\paragraph{Voordelen}
Gelijkaardig met schrijftelijke examens die deels op computer afgelegd worden dit is een heel erg veilig manier.

\paragraph{Nadelen}
Hier hetzelfde als schrijftelijke examens die deels op computer afgelegd worden, behoorlijk duur. Enkel is hier nog wat extra werk nodig om het ingevulde examen op elk systeem af te halen.

\subsubsection{Schriftelijke examens (enkel via een webbrowser)}

\paragraph{Voordelen}
Deze kunnen zowel op computers van HoGent als op eigen devices van de student afgenomen worden. 

\paragraph{Nadelen}
Wanneer deze examens op een device van de student afgenomen worden, is de enigste controle die er gevoerd kan worden een visuele controle. Er wordt niet gemonitord voor webverkeer/onderlinge communicatie gekeken of de student documenten met antwoorden op zijn device heeft staan. 


\section{BYOD Devices}

BYOD is een term die je de laatste jaren wel vaker begint te horen. Door de overvloed van nieuwe apparaten en gadgets, die aan een rotvaart op de markt gebracht worden, is het voor vele bedrijven te om altijd mee te zijn met de meest actuele technologiën. De opkomst van BYOD ofwel Bring Your Own Device zorgt voor een verschuiving van de overheadkosten, die het beheren van vele apparaten in bezit van het bedrijf met zich meebrengen, weg van het bedrijf \autocite{Hong2016}.

%Je verwijst bij elke bewering die je doet, vakterm die je introduceert, enz. naar je bronnen. In \LaTeX{} kan dat met het commando \texttt{$\backslash${textcite\{\}}} of \texttt{$\backslash${autocite\{\}}}. Als argument van het commando geef je de ``sleutel'' van een ``record'' in een bibliografische databank in het Bib\TeX{}-formaat (een tekstbestand). Als je expliciet naar de auteur verwijst in de zin, gebruik je \texttt{$\backslash${}textcite\{\}}.
%Soms wil je de auteur niet expliciet vernoemen, dan gebruik je \texttt{$\backslash${}autocite\{\}}. In de volgende paragraaf een voorbeeld van elk.

%\textcite{Knuth1998} schreef een van de standaardwerken over sorteer- en zoekalgoritmen. Experten zijn het erover eens dat cloud computing een interessante opportuniteit vormen, zowel voor gebruikers als voor dienstverleners op vlak van informatietechnologie~\autocite{Creeger2009}.




%%=============================================================================
%% Methodologie
%%=============================================================================

\chapter{Methodologie}
\label{ch:methodologie}

%% TODO: Hoe ben je te werk gegaan? Verdeel je onderzoek in grote fasen, en
%% licht in elke fase toe welke stappen je gevolgd hebt. Verantwoord waarom je
%% op deze manier te werk gegaan bent. Je moet kunnen aantonen dat je de best
%% mogelijke manier toegepast hebt om een antwoord te vinden op de
%% onderzoeksvraag.


\section{Voorbereiding van het onderzoek}
In het begin van dit onderzoek is er een scope vastgelegd in samenspraak met Bert Van Vreckem. Namelijk dat er in geen enkel geval software op de laptops ge\"{\i}nstalleerd mag worden voor het BYOD systeem. Toen bracht hij de suggestie van een beveiligde netwerkomgeving reeds op en gaf hier duiding bij. De opstelling van de beveiligde omgeving is in dit onderzoek gebaseerd op mondelinge feedback door Bert Van Vreckem. Later stelde meneer Van Vreckem de andere BYOD systemen ook nog voor, deze zijn ook opgenomen in het onderzoek.

\section{Onderzoek}


De eerstvolgende stap was onderzoeken wat het huidige systeem is, en op welke plaatsen er verbeteringen nodig zijn. Daarvoor is er een interview opgesteld. Dit interview is bij 2 docenten en 1  systeembeheerder afgenomen. Met de informatie die vergaard werd uit die interviews is beslist wat de optimale oplossing is voor de huidige probleemstelling.

Daarna is er onderzoek gedaan naar de opstelling van die beveiligde netwerkomgeving. Onderzoek naar Hardware, Software en hoe deze voor de proof of concept opgezet zou worden. Al snel bleek dat een gevirtualiseerde opstelling meer haalbaar was dan een fysieke omgeving. De gevirtualiseerde omgeving is opgezet met GNS3. 

\subsection{GNS3}
GNS3 ofwel Graphical Network Simulator 3 is een network software emulator. Deze software staat toe om virtuele en fysieke devices met elkaar te verbinden om zo complexe netwerken te simuleren. GNS3 kan virtuele machines van VMWare, VirtualBox of KVM gewoon importeren en routeren. Dankzij GNS3 was het testen van een opstelling tijdens het onderzoek een stuk makkelijker en kon het opstellen van een proof of concept versneld worden.

\subsubsection{Tekortkoming}
Wireless access points zijn niet ondersteund binnen GNS3. Voor de proof of concept is er dus gebruik gemaakt van een opstelling met wired connections. In de realiteit verkiezen we dus wireless access omdat niet alle moderne laptops nog een ethernetpport hebben.   

\subsection{Open-source software}
Omdat de opstelling open-source moest blijven is er voor de opstelling enkel naar open source netwerk operating systems/software gekeken. Onder andere:
\begin{itemize}	
	\item pfSense: Open-source operating system gebaseerd op FreeBSD. Router/firewall.
	\item OpenWrt: Open-source Linux operating system. Volledige netwerk oplossing. 
	\item dnsmasq: Open-source Unix-like operating system. DNS forwarder en DHCP Server.
\end{itemize}

\subsection{VMWare}
Om de virtualisatie van niet-netwerk apparaten, zoals end devices te faciliteren is er gebruik gemaakt van VMWare. Gebruik van Virtualbox of een KVM Server is ook mogelijk maar er is gekozen voor VMWare door personal preference. 

\subsection{Automatisatie}

Om het opstellen van dergelijke omgeving zo simpel mogelijk te maken heb ik ervoor gekozen om een groot stuk te automatiseren d.m.v. Ansible. Ansible is Agentless SSH automation. Dit wil zeggen dat er op de nodes die je beheert met ansible geen speciale software moeten draaien, enkel een OpenSSH server, deze is standaard geinstalleerd om de meeste systemen. Met Ansible en haar netwerk modules kan je makkelijk volledige netwerkomgeivngen opzetten zonder zelf aan de command-line interface te moeten komen. 

\section{Proof of concept}
Na het configureren van enkele opstellingen is er \'{e}\'{e}n uitgekozen. Degene die het makkelijkst was om op te zetten en het meest simpel om te  configureren. Daarover kan u in het volgende hoofdstuk lezen.

\chapter{Proof of concept}
\label{ch:proofofconcept}



\section{Opstelling}

De opstelling van de  proof of concept, die u terug kan vinden op \hyperref[fig:Poc1]{figuur 4.1}, bestaat uit:
\begin{itemize}
	\item pfSense router en firewall
	\item CentOS server met een dnsmasq DNS en DHCP Server
	\item CentOS server met Gitlab Community Edition ge\"{\i}nstalleerd.
	\item Switch
	\item End devices (laptops)
\end{itemize}

Opmerking: bij gebrek aan ondersteuning van open-source Wireless Access Points binnen GNS3, zijn de laptops van de studenten bedraad verbonden met het netwerk. Bij een echte opstelling is het de bedoeling dat er een Captive Portal geconfigureerd wordt op de pfSense Router zodat enkel studenten met toestemming kunnen verbinden. PfSense routers ondersteunen o.a. LDAP Authenticatie, waarmee je gebruikers met hun Microsoft Active Directory Account kan authenticeren om in te loggen op het netwerk.   
	
\begin{figure}
	\includegraphics[width=\linewidth]{img/gns3FinalPoC.jpg}
	\caption{Opstelling voorgesteld in GNS3}
	\label{fig:PoC1}
\end{figure}

\subsection{Functionaliteiten}

Hier volgt een opsomming van alle functionaliteiten die een dergelijke opstelling bevat.

\begin{itemize}
	\item Alle admin systemen zijn gehardened en beveiligd volgens een Baseline gebaseerd op de CIS Guidelines.
	\item De het opzetten van de omgeving voor een examen wordt grotendeels door Ansible gedaan.
	\item Het is simpel om de examenomgeving op te zetten voor een lector (dankzij Ansible). 
	\item De studenten kunnen enkel browsen naar websites die toegestaan zijn door de lecoren.
	\item Beveiliging op zowel de firewall en DNS zorgen ervoor dat studenten geen VPN-verbinding kunnen maken of een andere DNS Server kunnen gebruiken.
	\item Studenten kunnen hun examen van de lokale Gitlab server halen en na het examen indienen via Gitlab
	\item Er wordt gemonitord of studenten niet met andere netwerken verbinden.
\end{itemize}

\subsection{Ontbrekende functionaliteiten}

Deze proof of concept is geen ideale oplossing voor het nieuwe examensysteem, het is wel de meest ideale oplossing binnen dit onderzoek.
Toch blijft er een functionaliteit over die voor sommige examens een nice-to-have is en voor andere een must-have, namelijk: 

Studenten hebben volledige toegang tot documenten op hun eigen laptop. Indien zij reeds gemaakte oefeningen, opgeloste voorbeeldexamens of vorige versies van examens bij zouden hebben, is het mogelijk dat zij een oneerlijk voordeel krijgen tegenover studenten die dit niet bijhebben. Volgens een lector die programmeerexamens afneemt, maakt dat voor examens zoals Object-Geori\"{e}nteerd Programmeren III minder uit vergeleken met Object-Geori\"{e}nteerd Programmeren I, waar het die lector noodzakelijk is dat studenten geen toegang hebben tot alle oefeningen, bijvoorbeeld.

\subsection{Automatisatie}
Automatisatie gebeurt met Ansible. Ansible communiceert met de servers via SSH met 4096bit SSH keys. Het is niet mogelijk om op die servers met een wachtwoord in te loggen.

\subsubsection{DNS \& Firewall}
Wanneer studenten toch websites mogen bezoeken zoals bijvoorbeeld de java api, moet er een aanpassing gemaakt worden in DNS en Firewall. Wanneer de lector de URL van een site ingeeft bij het uitvoeren van de ansible playbook, worden de juiste aanpassingen gemaakt op beide servers.

\subsection{Gitlab}
Gebruikers worden in deze proof-of-concept via de Gitlab API toegevoegd met Ansible. Wanneer de docent gebruikers aanlevert in een .csv bestand, dan leest Ansible die uit en voegt ze toe aan de Gitlab server. 

Dit is de Ansible taak die per student in die .csv file uitgevoerd wordt. De waarden die tussen: "\{\{\}\}"   staan, zijn variabelen die per student verschillend zijn.
\lstset{basicstyle=\ttfamily}
\begin{lstlisting}
-  name: Create Gitlab User
	gitlab_user:
		server_url: https://gitlabsrv01.benoitballiu.be
		validate_certs: True
		api_username: admin
		api_password: $encrypted$
		group: "{{ group }}"
		access_level: developer
		name: "{{ studentName }}"
		username: "{{ username }}"
		email: "{{ studentEmail }}"
		password: dummypassword
		state: present
	delegate_to: localhost
\end{lstlisting}

Het volledige Ansible Playbook kan u in de bijlagen terugvinden.

\section{Verloop van een examen}
Deze sectie geeft meer uitleg over BYOD-examen met deze opstelling, door een voorbeeld van het verloop te geven.

\subsection{Voorbeeld: Examen OOPII}

\begin{itemize}
	\item De lector bereidt de avond voor het examen of op de dag zelf het examen voor, door een een Ansible Playbook te doorlopen. Ansible zal zo users, studenten die het examen afleggen, toevoegen aan zowel de gitlabserver als de Captive portal van de router en, indien gewild, url's te whitelisten. Zie figuur \hyperref[fig:PoC2]{4.2}, dit is het scherm dat de lector invult. Zowel figuur \hyperref[fig:PoC3]{4.3} als figuur \hyperref[fig:PoC4]{4.4} voor een voorbeeld van de files die de lector moet invullen. 

	\begin{figure}[H]
		\includegraphics[width=\linewidth]{img/AdminPC.png}
		\caption[Voorbeeld van het Ansible Playbook]{Dit is hoe een docent een examen voorbereidt. een lector geeft hier de locaties van de gevraagde bestanden in.}
		\label{fig:PoC2}
	\end{figure}
	
	\begin{figure}[H]
	\includegraphics[width=\linewidth]{img/CSV1.png}
	\caption[Voorbeeld van Users.csv]{Boven zie je een voorbeeld van een User file die een lector invult, onder zie je dezelfde file op de manier dat Ansible hem inleest.}
	\label{fig:PoC3}
\end{figure}

	\begin{figure}[H]
	\includegraphics[width=\linewidth]{img/CSV2.png}
	\caption[Voorbeeld van Links.csv] {Boven zie je een voorbeeld van een Link file die een lector invult, onder zie je dezelfde file op de manier dat Ansible hem inleest.}
	\label{fig:PoC4}
\end{figure}

	\item De lector kan op de adminmachine testen of de studenten enkel aan de toegestane url's kunnen.
	
		\begin{figure}[H]
		\includegraphics[width=\linewidth]{img/Linkblock1.png}
		\caption[Screenshot van webbrowser 1] {Hierboven zie je een screenshot van wat er gebeurt wanneer de docent een site verschillend van de toegelaten sites uittest.}
		\label{fig:PoC5}
	\end{figure}

		\begin{figure}[H]
		\includegraphics[width=\linewidth]{img/Linkblock2.png}
		\caption[Screenshot van webbrowser 2] {Hierboven zie je een screenshot van wat er gebeurt wanneer de docent één van de toegestane sites uittest.}
		\label{fig:PoC6}
	\end{figure}
	\item De lector voegt zijn project toe aan Gitlab. Dit kan hij manueel doen: 
			\begin{figure}[H]
		\includegraphics[width=\linewidth]{img/Git01.png}
		\caption[Screenshot van het aanmaken van een git repository] {Hierboven zie je een screenshot waar een lector een nieuw repository toegvoegt aan de gitlabserver.}
		\label{fig:PoC7}
	\end{figure}
    Gezien het thema van automatisatie is dit in het eindelijke playbook toch geautomatiseerd. 
   			\begin{figure}[H]
   	\includegraphics[width=\linewidth]{img/Git02.png}
   	\caption[Screenshot van het aanamken van een git repository met Ansible] {Hierboven zie je hoe het aanmaken van een gitlabproces geautomatiseerd is.}
   	\label{fig:PoC8}
   \end{figure} 
	\item Wanneer de studenten binnenkomen forken ze hun eigen private repository van het examenproject.
	
	   			\begin{figure}[H]
		\includegraphics[width=\linewidth]{img/Fork1.png}
		\caption[Screenshot van een forked repository] {Hierboven zie je hoe de student een fork van de starterscode kan maken.}
		\label{fig:PoC9}
	\end{figure} 
	
	\item Tijdens het examen kunnen de studenten reeds pushen naar hun repository. Op het einde van het examen controleert de student of zijn oplossing correct op github staat.
	\item De Lector kan nu alle repositories van de studenten binnenhalen en verbeteren.
	   			\begin{figure}[H]
		\includegraphics[width=\linewidth]{img/Check01.png}
		\caption[Screenshot van het overzicht die een lector heeft.] {Hierboven zie je wat de lector ziet na een examen, in dit geval heeft slechts 1 student een fork gemaakt}
		\label{fig:PoC10}
	\end{figure} 
	
	
\end{itemize}






% Voeg hier je eigen hoofdstukken toe die de ``corpus'' van je bachelorproef
% vormen. De structuur en titels hangen af van je eigen onderzoek. Je kan bv.
% elke fase in je onderzoek in een apart hoofdstuk bespreken.

%\input{...}
%\input{...}
%...

%%=============================================================================
%% Conclusie
%%=============================================================================

\chapter{Conclusie}
\label{ch:conclusie}

%% TODO: Trek een duidelijke conclusie, in de vorm van een antwoord op de
%% onderzoeksvra(a)g(en). Wat was jouw bijdrage aan het onderzoeksdomein en
%% hoe biedt dit meerwaarde aan het vakgebied/doelgroep? Reflecteer kritisch
%% over het resultaat. Had je deze uitkomst verwacht? Zijn er zaken die nog
%% niet duidelijk zijn? Heeft het onderzoek geleid tot nieuwe vragen die
%% uitnodigen tot verder onderzoek?


In dit onderzoek is gezocht naar een antwoord op de vraag: 'Is het mogelijk een omgeving, die beveiligd is en voldoet aan de eisen van Hogeschool Gent, op te stellen waar studenten computerexamens op hun eigen laptop kunnen afleggen?'

Uit het onderzoek naar een nieuwe, BYOD, examenomgeving is gebleken dat dergelijke omgeving nooit genoeg beveiligd kan worden tegen fraude. Dit natuurlijk binnen de scope die aan het onderzoek gegeven is, het is namelijk niet de bedoeling dat er software op end devices ge\"{\i}nstalleerd wordt. De beveiligde netwerkomgeving, zoals voorgesteld in het hoofdstuk proof of concept, is de enigste omgeving die tegen fraude via het internet beschermt, maar in dergelijke omgeving is er geen controle over de documenten waar de student over beschikt. De nieuwe examenomgeving zou minstens het zelfde niveau van beveiliging als het huidige NetSupport School systeem moeten halen. Binnen de opgelegde scope is dit niet mogelijk. 


Het principe van bring-your-own-device examens is wel interessant gebleken, vooral door de lagere kosten en kleinere administratieve overhead. Het zou daarom interessant zijn om dit onderzoek te hernemen, met een bredere scope. Indien bij het onderzoeken van een BYOD examenomgeving, het installeren van (kiosk)software op end devices toegestaan is, zal dergelijk onderzoek een geheel ander resultaat opleveren.






%%=============================================================================
%% Bijlagen
%%=============================================================================

\appendix

%%---------- Onderzoeksvoorstel -----------------------------------------------

\chapter{Onderzoeksvoorstel}

Het onderwerp van deze bachelorproef is gebaseerd op een onderzoeksvoorstel dat vooraf werd beoordeeld door de promotor. Dat voorstel is opgenomen in deze bijlage.

% Verwijzing naar het bestand met de inhoud van het onderzoeksvoorstel
%---------- Inleiding ---------------------------------------------------------

\section{Introductie} % The \section*{} command stops section numbering
\label{sec:introductie}

 Momenteel worden de meeste computerexamens op Hogeschool Gent in een beveiligde omgeving op een computer van de hogeschool afgenomen. Deze methodiek zorgt voor een hoge kost. De hogeschool moet een groot aantal computers te beschikking stellen. Deze moeten beschikken over software eigen aan het examen, en monitoringsoftware bevatten zodat examens in een beveiligde omgeving afgelegd kunnen worden. \\ In deze bachelorproef onderzoeken we de haalbaarheid van een beveiligde omgeving waar studenten computerexamens op hun eigen laptop kunnen afleggen. Deze beveilige omgeving moet maximaal vermijden dat fraude mogelijk is. \\

Deelonderzoeksvragen:
  \begin{itemize}
     \item Hoe gebeuren computerexamens nu op de hogeschool? Welke tools worden gebruikt? Welke regels en beperkingen leggen lectoren nu typisch op bij computerexamens?
     
     \item Welke tools/oplossingen bestaan er tegenwoordig voor dit soort situaties? Zijn die voldoende flexibel om bijvoorbeeld een examen programmeren of andere ict-vakken te faciliteren?
     
     \item Als we zelf een omgeving willen opzetten, welke componenten moet die dan bevatten? Welke beperkingen kunnen we studenten opleggen en welk gedrag kunnen we niet vermijden?
  \end{itemize}

%Hier introduceer je werk. Je hoeft hier nog niet te technisch te gaan.
%Je beschrijft zeker:
%\begin{itemize}
%  \item de probleemstelling en context
%  \item de motivatie en relevantie voor het onderzoek
%  \item de doelstelling en onderzoeksvraag/-vragen
%\end{itemize}

%---------- Stand van zaken ---------------------------------------------------

\section{Literatuurstudie}
\label{sec:literatuurstudie}

\subsection{Voordelen van bring-your-own-device examens}

Studenten die een examen afleggen op hun eigen hardware zijn vertrouwd met hun toestel. Waardoor hun stressgevoeligheid afneemt, volgens
 \textcite{TeckSwee2014}. In dat onderzoek was er op een populatie van 672 studenten 74.02\% blij met deze manier om examens af te nemen. 

\subsection{Problemen bij bring-your-own-device examens}
\subsubsection{Fraude}
Aangezien examens afleggen op eigen hardware een relatief nieuw gegeven is, ervaren we nog enkele kinderziektes.\\ \textcite{Dawson2016} bekeek in zijn onderzoek 5 manieren om fraude te plegen tijdens een examen op eigen laptop, namelijk: \\
\begin{itemize}
    \item De examenopgave lokaal opslaan en achteraf online plaatsen.
    \item Het examen op een virtuele omgeving afleggen en op die manier het besturingssysteem manipuleren
    \item Scripts uitvoeren via usb-keyboard hacks
    \item Software aanpassingen maken 
    \item Cold-boot aanval\\
\end{itemize}


Volgens het onderzoek van \textcite{VegendiaSindre2015} is het vermijden van elektronische communicatie een prioriteit. Dit is dan ook iets wat in de proof-of-concept onmogelijk gemaakt moet worden. 

\subsubsection{Overige problemen}

Naast fraude zijn er andere aandachtspunten bij het afnemen van examens op eigen hardware, \textcite{Hillier2015} heeft het in zijn onderzoek over onder andere: laptops die niet sterk genoeg zijn om bepaalde software aan te kunnen, onverwachte crashes van software of besturingssystemen, hardware die tijdens het examen faalt en batterijcapaciteit (indien er geen toegang tot stroom is). 



% Voor literatuurverwijzingen zijn er twee belangrijke commando's:
% \autocite{KEY} => (Auteur, jaartal) Gebruik dit als de naam van de auteur
%   geen onderdeel is van de zin.
% \textcite{KEY} => Auteur (jaartal)  Gebruik dit als de auteursnaam wel een
%   functie heeft in de zin (bv. ``Uit onderzoek door Doll & Hill (1954) bleek
%   ...'')


%---------- Methodologie ------------------------------------------------------
\section{Methodologie}
\label{sec:methodologie}

Voor dit onderzoek maken we een Analyse en beslissen we over de scope en de vereisten van de beveiligde omgeving. We gaan een onderzoek doen naar huidige oplossingen voor bring-your-own-device examens. \\ Via een rondvraag bij docenten gaan we onderzoeken welke vereisten elk computerexamen heeft. Zo kunnen we in de proof-of-concept een preset maken voor elk specifiek examen op computer. \\ De proof-of-concept zelf zal een volledig virtuele omgeving zijn, daarvoor gaan we o.a. gebruik maken van GNS3 en VirtualBox. 

%---------- Verwachte resultaten ----------------------------------------------
\section{Verwachte resultaten}
\label{sec:verwachte_resultaten}

Aan het einde van dit onderzoek verwachten we een compleet werkende en veilige proof-of-concept omgeving te hebben. De bedoeling is dat docenten hiermee vlot hun examens kunnen voorbereiden en verzekerd zijn van de fraudebestendigheid.


%---------- Verwachte conclusies ----------------------------------------------
\section{Verwachte conclusies}
\label{sec:verwachte_conclusies}

We verwachten dat een dergelijke omgeving op te zetten is. Al vrezen we wel dat er enkele beperkingen aan dit systeem zullen zijn. Bijvoorbeeld: aangezien het niet de bedoeling is dat er software op de laptop van een student geïnstalleerd wordt, zal je extra informatie die een student op zijn laptop zelf staan heeft niet kunnen verbieden. 




%\addcontentsline{toc}{chapter}{\textcolor{maincolor}{\IfLanguageName{dutch}{Bibliografie}{Bibliography}}}
%%---------- Andere bijlagen --------------------------------------------------

\chapter{Interviews}

Extra informatie werd verworven doormiddel van interviews met Docenten op Hogeschool Gent, Faculteit Bedrijf en Organisatie.


\section{Orientatie}

Momenteel wordt er naar 4 systemen gekeken voor dit onderzoek:
\begin{itemize}
	\item Safe Exam Browser 
	\item Televic AssessmentQ / AVIDAnet Lite
	\item Desktop virtualisatie via een cloudprovider
	\item Beveiligde, configureerbare netwerkomgeving om internettoegang tot niet-toegestane sites vanop laptops te vermijden
\end{itemize}
De vragen die gesteld worden tijdens dit interview zijn bedoeld om af te toetsen welk systeem het meest geschikt blijkt volgens docenten. 

\section{Vragen}

\begin{itemize}
	\item Wat vindt u de beste functies van het huidige systeem om examens op de computers van HoGent af te leggen?
	\bigskip 
	\item Welke functies ontbreken momenteel aan het huidige systeem om examens op de computers van HoGent af te leggen?
	\bigskip 
	\item Hoelang duurt het momenteel om een examen op computer voor te bereiden en hoe wordt dit gedaan (Computerlokaal voorbereiden, juiste examenfiles op de computer plaatsen) en hoe lang duurt het achteraf om alle examens op te halen?
	\bigskip 
	\item Wat vindt u het meest belangrijk? De student die niet kan communiceren via digitale kanalen en enkel op sites kan die toegestaan zijn voor het examen, of De student die niet aan files op zijn eigen systeem kan. Of vindt u beiden even belangrijk?
	\bigskip 
	\item Hoelang zou u maximaal willen bezig zijn met de voorbereiding van een examen?
	\bigskip 
	\item Heeft u nog vragen, Ideeën of opmerkingen over het nieuwe systeem?
	
\end{itemize}



\chapter{Security Baseline}

Deze Baseline is in samenwerking met Jan De Nul opgesteld en is toegevoegd ter verduidelijking.

\section{Security Baseline: Proposal}
This proposal is based on the CIS guidelines and extensive research on Information Security. 

Athors: Benoit Balliu, Dave Thyssen [Jan De Nul]

\subsection{Secure Shell Hardening}
Since we are working in a virtualized environment, no machines are accessed physically. That is why our only secure way to reach the systems, SSH, should be hardened. 
It is generally preferred to use SSH key pairs to login to the systems. 
Both keys and passwords have their pros and cons. SSH keys are long and complex, far more than any password could be. Passwords are generally, predictably, unavoidably weak. While it is possible to have strong passwords, it has been shown that people will use weak passwords and have poor password practices... short, simple, word-based, simple patterns ("p@ssw0rd!"), write them down, use them on multiple sites.
Bad passwords aside, even "good" passwords are vulnerable to brute-force (online or offline) under the right conditions.
The balance of evidence strongly suggests that passwords are weaker and keys are stronger.

\subsection{Kernel Hardening}
One of the most common things SysAdmins forget to harden is the Linux kernel, since Linux actually gives people a false sense of security. Extra tweaks should be done to the Linux kernel. 

\subsection{Firewall Hardening}
Even though most Datacenters are already protected by hardware firewalls, software firewalls still have to be secure, to mitigate threats from the inside. The firewall should have a default inbound drop policy. By default only SSH should be allowed. 

\subsection{Banner Policy}
Though it might not seem as important, a custom banner hides sensitive system information and scares off some potential attackers. 

\subsection{Fail2Ban}
Fail2Ban is a great and easily customizable piece of software to reduce the amount of times someone can login and ban them after a certain amount of tries (Forever or for a given amount of time). Configuration happens in a jail.local file. 

\subsection{Password Policy}

Password expiry is outdated. Many big security advisories is advising against it: ...In this day and age, changing passwords every 90 days gives you the ILLUSION of stronger security while inflicting needless pain and cost to your organization... (SANS.org)

After asking a question to a board of infosec experts, they were able to give me the following answer:
If your passwords are strong (randomly generated by password managers) and one still manged to leak and find its way into the hands of an attacker, the solution is to fix the source of the leak, not to blindly update the passwords. Changing the passwords will not prevent new passwords to leak in the same way. It would be wiser to prevent password guessing attempts and to try detecting abnormal authentications. Using proper 2FA could also better mitigate passwords leaks.


\subsection{Default Umask}
This script changes the default permission from Permissive to Moderate. As recommended by CIS. 

\subsection{Blacklisting}
Unused and unsafe or deprecated modules/services should be blacklisted. The blacklist.conf file blacklists all the modules/services as recommended by CIS.


\subsection{SELinux Bolean values}
A SysAdmin should \textbf{never} disable SELinux, even though some sysadmins hate it. 
SELinux implements Mandatory Access Control (MAC). Every process and system resource has a special security label called a SELinux context. A SELinux context is an identifier, which abstracts away the system-level details and focuses on the security properties of the entity. Not only does this provide a consistent way of referencing objects in the SELinux policy, but it also removes any ambiguity that can be found in other identification methods; for example, a file can have multiple valid path names on a system that makes use of bind mounts. 





\subsection{Security Through Obscurity}
“Relying upon security through obscurity is bad: it usually leads to fragile or insecure systems”.
Some of the configuration rules I have implemented are just obfuscation rather than hardening. Many Info Security Experts believe that this is not a valid way of securing things, which is true, but it does add another layer of security. An extra layer of obscurity might get rid of Script-kiddies or less experienced hackers. Even experienced hackers will have to make more noise to get more information, which makes them more vulnerable to detection (IDS). 

Obscurity might add some additional security, but you should not rely upon it, and it should not be your primary defence. You should be prepared that the obscurity might be pierced, and be confident that you have adequate defences to handle that case.
This is a very controversial subject in the Cybersecurity world. After extensively reading the points that both parties make, I believe obscurity to be nothing more or less than an extra layer of security. Which is nothing bad in itself. The difficulties that more obscurity brings to the table are not of that much importance in this organisation, since only a select group of people will have to deal with these difficulties. 


\subsection{Unaccepted Best Practices}
This is a list of best practices I chose not to implement and why. 
\subsubsection{Disabling the root user}
I decided to make sure that people cannot login as the root user, but can still change to the root user when inside the system. Since experienced SysAdmins alone will use the system’s cli, I believe that we do not have to disable the root user as a whole. 

\subsubsection{Automatic  Security Updates}
After talking to my supervisors, we decided not to include any automatic patches/updates. This does increase the risk of a system becoming out of date or vulnerable. On the other hand, there is no risk that a production environment will crash because of a patch/update. I decided not to include the configuration files for this. Spacewalk will handle these upgrades (for now) so the systems will be kept up to date. We should look into automation for this. 

\chapter{Ansible Playbooks}

TODO: Toevoegen van Ansible yaml files met yml2tex https://github.com/megrxu/YAML-to-TeX

\input{ansible}

%%---------- Referentielijst --------------------------------------------------

\printbibliography[heading=bibintoc]

\end{document}
