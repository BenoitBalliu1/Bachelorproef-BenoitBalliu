\chapter{Stand van zaken}
\label{ch:stand-van-zaken}

% Tip: Begin elk hoofdstuk met een paragraaf inleiding die beschrijft hoe
% dit hoofdstuk past binnen het geheel van de bachelorproef. Geef in het
% bijzonder aan wat de link is met het vorige en volgende hoofdstuk.

% Pas na deze inleidende paragraaf komt de eerste sectiehoofding.

Voor er onderzoek gedaan wordt naar oplossing voor examens op eigen computer moet er gekeken worden naar de huidige situatie op de Hogeschool Gent en hoe andere instellingen deze probleemstelling aanpakken. 
%Dit hoofdstuk bevat je literatuurstudie. De inhoud gaat verder op de inleiding, maar zal het onderwerp van de bachelorproef *diepgaand* uitspitten. De bedoeling is dat de lezer na lezing van dit hoofdstuk helemaal op de hoogte is van de huidige stand van zaken (state-of-the-art) in het onderzoeksdomein. Iemand die niet vertrouwd is met het onderwerp, weet er nu voldoende om de rest van het verhaal te kunnen volgen, zonder dat die er nog andere informatie moet over opzoeken \autocite{Pollefliet2011}.

\section{Examens op Hogeschool Gent}

\subsection{Soorten examens}
Op Hogent, Faculteit Bedrijf en Organisatie worden er 2 soorten examens afgenomen, mondelinge en schriftelijke. Enkele schriftelijke examens worden enkel op papier afgelegd, anderen deels of volledig op computer. Dit zijn de soorten computerexamens die op HoGent afgenomen worden.
\begin{itemize}
\item Een online test waarbij antwoorden via de webbrowser invgevuld worden (bv. via Chamilo, Cisco-platform, enz.)
\item Een schriftelijk examen waaarbij de voorbereiding op pc gebeurt maar de antwoorden op papier ingevuld worden.
\item Een examen dat op computer m.b.v. specifieke software (bv. IDE en compiler) gemaakt wordt waarna het resultaat digitaal ingediend wordt (bv. Word document met antwoorden, zip-bestand met broncode, Github, ...)
\end{itemize}


\subsubsection{Schriftelijke examens (deels op computer) }

De student heeft toegang tot software (vb. Netbeans, Microsoft Excel, Microsoft Word) en documenten (vb. Examenopgave, Microsoft PowerPoints, PDF-documenten) die zich lokaal bevinden, op vraag van de lector. Deze examens worden altijd op desktops van Hogeschool Gent afgenomen. In een beveiligde gemonitorde omgeving, waarin je enkel kan wat toegestaan is door de docent. De examenopzichter heeft via de admin-computer zicht op alle bureaubladen van de studenten die het examen aan het afleggen zijn. 




\subsubsection{Schriftelijke examens (volledig op computer)}

\paragraph{Examens waarbij toegang tot het gehele systeem vereist is}

Met toegang tot het gehele systeem wordt er bedoeld dat de student hier ook toegang heeft tot software  en documenten die zich lokaal bevinden. Deze examens worden net zoals examens die deels op computer afgenomen worden, altijd op desktops van Hogeschool Gent afgenomen, in diezelfde beveiligde omgeving. Enkel moet de student zijn ingevulde examen digitaal indienen en worden eventuele notities op papier niet bekeken.  

\paragraph{Examens waarbij enkel toegang tot een webbrowser vereist is}

Wanneer er enkel toegang tot een webbrowser vereist is (Test op chamilo, Online-vragenlijst) dan hoeft het examen niet op een desktop van Hogeschool Gent afgelegd te worden. Dit kan ook gewoon via de browser op de laptop van een student. Dit brengt natuurlijk enkele risico's met zich mee, de lector heeft geen zicht op hetgeen de student doet en de student kan communiceren met medestudenten achter de rug van een opzichter. Er kunnen ook bepaalde problemen met de browser opduiken, zoals problemen met plugins en browsertype of browserversie. 

\section{BYOD}

BYOD ofwel Bring Your Own Device is een term die je de laatste jaren wel vaker begint te horen. Door de overvloed van nieuwe apparaten en gadgets, die aan een rotvaart op de markt gebracht worden, is het voor vele bedrijven te duur om altijd mee te zijn met de meest actuele technologiën. De opkomst van BYOD zorgt voor een verschuiving van de overheadkosten, die het beheren van vele apparaten in bezit van het bedrijf met zich meebrengen, weg van het bedrijf naar de werknemers toe \autocite{Hong2016}.

\section{BYOD examens}
Bij BYOD examens leggen studenten examens af op hun eigen laptop. Deze manier van werken is nog niet wijdverspreid. Het kent enkele voor- en nadelen, in de volgende secties zal ik hier wat verder over uitweiden. 

\subsection{Voordelen bij BYOD examens}


\subsection{Opmerkingen bij BYOD examens}

Aangezien examens afleggen op eigen hardware een relatief nieuw gegeven is, ervaren we nog enkele kinderziektes.die problemen kunnen we in 2 groepen indelen. Veiligheid en Werkbaarheid.\\ 

\subsubsection{Problemen met veiligheid}
Wanneer je niet de volledige controle over een dapparaat hebt kan de mogelijkheid tot fraude niet altijd even makkelijk of zelfs niet vermeden worden.
\textcite{Dawson2016} bekeek in zijn onderzoek 5 manieren om fraude te plegen tijdens een examen op eigen laptop, namelijk: 
\begin{itemize}
	\item De examenopgave lokaal opslaan en achteraf online plaatsen.
	\item Het examen op een virtuele omgeving afleggen en op die manier het besturingssysteem manipuleren
	\item Software aanpassingen maken 

\end{itemize}

Deze methodes zijn gerangschikt in de orde van de kans dat een student tijdens de beperkte examentijd \'{e}\'{e}n van deze methodes toepast. Eerst ga ik wat meer uiteg geven bij de 2 meest gangbare methodes: 

\textbf{De examenopgave lokaal opslaan en achteraf online plaatsen.} De student krijgt de examenopgave lokaal, niets weerhoudt hem dus om daar lokaal een kopie van te maken en om die achteraf (tegen betaling) te verspreiden. Dit zorgt er voor dat de meeste studenten de examens die het jaar ervoor gegeven zijn kunnen inkijken en oplossen of de oplossingen kunnen bekijken. In de snel veranderende wereld van Information Technology hoeft dit geen probleem te zijn, de leerstof kan per jaar vari\"{e}ren. Het betekent natuurlijk wel dat de student de vraagstelling of soort oefeningen die op het examen zal voorkomen beter kan inschatten. Met dit gegeven zal rekeninng gehouden moeten worden bij het opstellen van examens die afgenomen worden op devices van de student. 

\textbf{Het examen op een virtuele omgeving afleggen en op die manier het besturingssysteem manipuleren.} Wanneer er tijdens een computerexamen gemonitord zou worden op systeemvlak, kan een student makkelijk die monitoringsoftware installeren in een virtuele machine en dan op het hostsysteem niet gemonitorde acties uitvoeren (Opzoeken op het internet, communicatie via het internet, opzoeken in lokale documenten). Hierop kan enkel gecontroleerd worden door manueel elke laptop af te gaan om te kijken of de software op de host geinstalleerd is, maar zelfs deze methode is niet waterdicht. Je kan enkel zeker zijn dat de student niet met een virtuele machine werkt wanneer de instelling zelf de software zou installerent. Maar, als devices dan toch beheerd worden door de instelling zelf verlies je de voordelen van het BYOD-principe (minder overhead en minder kosten).

\textbf{software aanpassingen maken.} Aangezien de student de software zelf beheert heeft hij de kans op de software aan te passen, zeker wanneer het open-source software is. Zoals theorie of formules verstoppen in source code. De reden waarom dit zo laag op de lijst staat is omdat het voor een student veel makkelijker is om gewoon een cheat sheet als document te verstoppen op zijn laptop en zo theorie of formules te bekijken. 

\subsubsection{Problemen met werkbaarheid}
Naast fraude zijn er andere aandachtspunten bij het afnemen van examens op eigen hardware, \textcite{Hillier2015} heeft het in zijn onderzoek over onder andere: laptops die niet sterk genoeg zijn om bepaalde software aan te kunnen, onverwachte crashes van software of besturingssystemen, hardware die tijdens het examen faalt en batterijcapaciteit (indien er geen toegang tot stroom is). Enkele van deze problemen zouden vermeden kunnen worden door minimum hardwarerequirements op te stellen voor laptops en om ervoor te zorgen dat examens altijd in een lokaal met toegang tot netstroom afgenomen worden. Maar zelfs dan kan er niet met 100\% zekerheid gezegd worden dat er geen hardware of software faalt tijdens die examens. Hetzelfde kan natuurlijk gezegd worden voor desktops van de instelling zelf, maar dan ligt de verantwoordelijkheid bij de instelling, niet bij de student. 


    


%Je verwijst bij elke bewering die je doet, vakterm die je introduceert, enz. naar je bronnen. In \LaTeX{} kan dat met het commando \texttt{$\backslash${textcite\{\}}} of \texttt{$\backslash${autocite\{\}}}. Als argument van het commando geef je de ``sleutel'' van een ``record'' in een bibliografische databank in het Bib\TeX{}-formaat (een tekstbestand). Als je expliciet naar de auteur verwijst in de zin, gebruik je \texttt{$\backslash${}textcite\{\}}.
%Soms wil je de auteur niet expliciet vernoemen, dan gebruik je \texttt{$\backslash${}autocite\{\}}. In de volgende paragraaf een voorbeeld van elk.

%\textcite{Knuth1998} schreef een van de standaardwerken over sorteer- en zoekalgoritmen. Experten zijn het erover eens dat cloud computing een interessante opportuniteit vormen, zowel voor gebruikers als voor dienstverleners op vlak van informatietechnologie~\autocite{Creeger2009}.



