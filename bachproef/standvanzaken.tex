\chapter{Stand van zaken}
\label{ch:stand-van-zaken}

% Tip: Begin elk hoofdstuk met een paragraaf inleiding die beschrijft hoe
% dit hoofdstuk past binnen het geheel van de bachelorproef. Geef in het
% bijzonder aan wat de link is met het vorige en volgende hoofdstuk.

% Pas na deze inleidende paragraaf komt de eerste sectiehoofding.

Voor er onderzoek gedaan wordt naar oplossing voor examens op eigen computer moet er gekeken worden naar de huidige situatie op de Hogeschool Gent en hoe andere instellingen deze probleemstelling aanpakken. 
%Dit hoofdstuk bevat je literatuurstudie. De inhoud gaat verder op de inleiding, maar zal het onderwerp van de bachelorproef *diepgaand* uitspitten. De bedoeling is dat de lezer na lezing van dit hoofdstuk helemaal op de hoogte is van de huidige stand van zaken (state-of-the-art) in het onderzoeksdomein. Iemand die niet vertrouwd is met het onderwerp, weet er nu voldoende om de rest van het verhaal te kunnen volgen, zonder dat die er nog andere informatie moet over opzoeken \autocite{Pollefliet2011}.

\section{Examens op Hogeschool Gent}

\subsection{Soorten examens}
Op Hogent, Faculteit Bedrijf en Organisatie worden er 2 soorten examens afgenomen, mondelinge en schriftelijke. Enkele schriftelijke examens worden enkel op papier afgelegd, anderen deels of volledig op computer. Wanneer ik een schriftelijk examen dat gedeeltelijk op computer afgenomen wordt vernoem, dan doel ik op een examen waarbij je gebruik mag maken van software of documenten (online of offline) op computer, maar je toch een papieren versie indient, zoals: Partim Financieel Management (Jaar 1, semester 2). Een examen dat volledig op computer afgelegd wordt, zoals: OO Programmeren II (Jaar 1, semester 2), is een examen waar je gebruik kan maken van software en resources (online of offline) en je het examen digitaal invult. Deze 2 soorten 

\subsubsection{Schriftelijke examens (gedeeltelijk op computer) }

Deze examens worden altijd op desktops van HoGent afgenomen. In een beveiligde gemonitorde omgeving, waarin je enkel kan wat toegestaan is door de docent. De examenopzichter heeft via de admin-computer zicht op alle bureaubladen van de studenten die het examen aan het afleggen zijn. 

\paragraph{Voordelen}
Dit is een heel erg veilige manier om een gedeeltelijk computerexamen af te leggen. De student kan via die computer niets doen wat hij tijdens het examen niet mag doen. Aangezien het examen nog steeds op papier ingediend wordt is er ook geen extra werk voor de opzichter. Deze manier beschermt natuurlijk niet tegen een student die een spiekbriefje bij zich heeft of op zijn telefoon enkele dingen opzoekt, maar die controle moet door de examenopzichter uitgevoerd worden en in dit onderzoek wordt daarmee ook geen rekening gehouden.

\paragraph{Nadelen}
Deze manier kost handenvol geld. Per computer moet er een licentie betaald worden. Aangezien er een groot aantal examens tergelijkertijd afgelegd worden moet er ook een groot aantal computers over een dergelijke licentie beschikken.


\subsubsection{Schriftelijke examens (volledig op computer)}

\paragraph{Voordelen}
Gelijkaardig met schrijftelijke examens die deels op computer afgelegd worden dit is een heel erg veilig manier.

\paragraph{Nadelen}
Zelfde als schrijftelijke examens die deels op computer afgelegd worden, behoorlijk duur. Enkel is hier nog wat extra werk nodig om het ingevulde examen op elk systeem af te halen.


\section{BYOD Examens}

Het huidige systeem is verouderd en niet meer aan te houden volgens de systeembeheerders op HoGent, over alle richtingen heen zijn er zo'n 180 softwarepaketten. Ze zouden graag evolueren naar een BYOD-Systeem.

%Je verwijst bij elke bewering die je doet, vakterm die je introduceert, enz. naar je bronnen. In \LaTeX{} kan dat met het commando \texttt{$\backslash${textcite\{\}}} of \texttt{$\backslash${autocite\{\}}}. Als argument van het commando geef je de ``sleutel'' van een ``record'' in een bibliografische databank in het Bib\TeX{}-formaat (een tekstbestand). Als je expliciet naar de auteur verwijst in de zin, gebruik je \texttt{$\backslash${}textcite\{\}}.
%Soms wil je de auteur niet expliciet vernoemen, dan gebruik je \texttt{$\backslash${}autocite\{\}}. In de volgende paragraaf een voorbeeld van elk.

%\textcite{Knuth1998} schreef een van de standaardwerken over sorteer- en zoekalgoritmen. Experten zijn het erover eens dat cloud computing een interessante opportuniteit vormen, zowel voor gebruikers als voor dienstverleners op vlak van informatietechnologie~\autocite{Creeger2009}.



