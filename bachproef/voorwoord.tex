%%=============================================================================
%% Voorwoord
%%=============================================================================

\chapter*{Woord vooraf}
\label{ch:voorwoord}

%% TODO:
%% Het voorwoord is het enige deel van de bachelorproef waar je vanuit je
%% eigen standpunt (``ik-vorm'') mag schrijven. Je kan hier bv. motiveren
%% waarom jij het onderwerp wil bespreken.
%% Vergeet ook niet te bedanken wie je geholpen/gesteund/... heeft

Voor u ligt de bachelorproef: 'Analyse, architectuur en proof-of-concept van een beveiligde omgeving voor het afnemen van computerexamens op eigen laptop.' Het onderzoek hiervoor is in opdracht van Hogeschool Gent gevoerd. Deze bachelorproef is geschreven in het kader van mijn afstuderen aan de opleiding Toegepaste Informatica met afstudeerrichting Systeem- en Netwerkbeheer aan de Hogeschool Gent.

De deelonderzoeksvragen van dit onderzoek zijn opgesteld door mijn promotor en co-promotor Bert Van Vreckem. Bij deze wil ik hem graag bedanken voor de samenwerking en het extra inzicht binnen het examensysteem van Hogeschool Gent dat hij mij aanleverde. 

Verder wil ik een lector die computerexamens afneemt op Hogeschool Gent en een systeembeheerder die bij Hogeschool Gent werkt bedanken voor hun tijd, dankzij hun heb ik de pijnpunten van het huidige systeem en de vereisten van het nieuwe systeem in kaart kunnen brengen.

Tot slot wil ik graag mijn ouders en mijn vriendin bedanken, zij hebben me tijdens dit onderzoek moreel ondersteund en mijn bachelorproef helpen nalezen.

Ik wens u veel leesplezier toe.

Benoit Balliu \\
Gent, 24 mei 2019 