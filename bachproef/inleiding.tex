%%=============================================================================
%% Inleiding
%%=============================================================================

\chapter{Inleiding}
\label{ch:inleiding}



 Momenteel worden de meeste computerexamens op Hogeschool Gent in een beveiligde omgeving op een computer van de hogeschool afgenomen. Deze methodiek heeft enkele nadelen. De voornaamste is de onbetrouwbaarheid van de omgeving. Wanneer een systeem zo'n kritieke functie heeft, is er slechts een kleine foutenmarge, het huidige systeem overschrijdt dit marge.  Verder zorgt het systeem ook voor een hoge kost, de hogeschool moet een groot aantal computers ter beschikking stellen en deze moeten beschikken over software eigen aan het examen, en monitoringsoftware zodat de examens in een beveiligde omgeving afgelegd kunnen worden. \\ In deze bachelorproef onderzoeken we de haalbaarheid van een beveiligde omgeving waar studenten computerexamens op hun eigen laptop kunnen afleggen. Deze beveilige omgeving moet fraude uitsluiten. \\


\section{Probleemstelling}
\label{sec:probleemstelling}

Deze bachelorproef is in opdracht van Hogeschool Gent zelf. Het huidige systeem om computerexamens af te nemen op de Hogeschool beschikt niet over de nodige vereisten, verder is het ook te duur. Het zorgt voor een grote overhead voor de systeembeheerders en te veel werk voor de docenten zelf. Een systeem dat effici\"{e}nter en meer budgetvriendelijk is dient het huidige systeem te vervangen.



\section{Onderzoeksvraag}
\label{sec:onderzoeksvraag}
Is het mogelijk een omgeving, die beveiligd is en voldoet aan de eisen van Hogeschool Gent, op te stellen waar studenten computerexamens op hun eigen laptop kunnen afleggen? 

\subsection{Deelonderzoeksvragen}
\begin{itemize}
	 \item Hoe gebeuren computerexamens nu op de hogeschool? Welke tools worden gebruikt? Welke regels en beperkingen leggen lectoren nu typisch op bij computerexamens?
	
	\item Welke tools/oplossingen bestaan er tegenwoordig voor dit soort situaties? Zijn die voldoende flexibel om bijvoorbeeld een examen programmeren of andere ict-vakken te faciliteren?
	
	\item Als we zelf een omgeving willen opzetten, welke componenten moet die dan bevatten? Welke beperkingen kunnen we studenten opleggen en welk gedrag kunnen we niet vermijden?

\end{itemize} 

\section{Onderzoeksdoelstelling}
\label{sec:onderzoeksdoelstelling}

Het doel van dit onderzoek is om een proof-of-concept op te stellen die voldoet aan de eisen van de systeembeheerders en de docenten en die een verbetering is tegenover het huidige systeem qua gebruiksgemak en budget. 

\section{Opzet van deze bachelorproef}
\label{sec:opzet-bachelorproef}

% Het is gebruikelijk aan het einde van de inleiding een overzicht te
% geven van de opbouw van de rest van de tekst. Deze sectie bevat al een aanzet
% die je kan aanvullen/aanpassen in functie van je eigen tekst.

De rest van deze bachelorproef is als volgt opgebouwd:

In Hoofdstuk~\ref{ch:stand-van-zaken} wordt een overzicht gegeven van de stand van zaken binnen het onderzoeksdomein, op basis van een literatuurstudie.

In Hoofdstuk~\ref{ch:methodologie} wordt de methodologie toegelicht en worden de gebruikte onderzoekstechnieken besproken om een antwoord te kunnen formuleren op de onderzoeksvragen.


In hoofdstuk~\ref{ch:proofofconcept} wordt de finale opstelling van het voorstel voor een bring-your-own-device systeem voorgesteld.  

In Hoofdstuk~\ref{ch:conclusie}, tenslotte, wordt de conclusie gegeven en een antwoord geformuleerd op de onderzoeksvragen. Daarbij wordt ook een aanzet gegeven voor toekomstig onderzoek binnen dit domein.

