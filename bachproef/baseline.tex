\section{Security Baseline: Proposal}
This proposal is based on the CIS guidelines and extensive research on Information Security. 

Athors: Benoit Balliu, Dave Thyssen [Jan De Nul]

\subsection{Secure Shell Hardening}
Since we are working in a virtualized environment, no machines are accessed physically. That is why our only secure way to reach the systems, SSH, should be hardened. 
It is generally preferred to use SSH key pairs to login to the systems. 
Both keys and passwords have their pros and cons. SSH keys are long and complex, far more than any password could be. Passwords are generally, predictably, unavoidably weak. While it is possible to have strong passwords, it has been shown that people will use weak passwords and have poor password practices... short, simple, word-based, simple patterns ("p@ssw0rd!"), write them down, use them on multiple sites.
Bad passwords aside, even "good" passwords are vulnerable to brute-force (online or offline) under the right conditions.
The balance of evidence strongly suggests that passwords are weaker and keys are stronger.

\subsection{Kernel Hardening}
One of the most common things SysAdmins forget to harden is the Linux kernel, since Linux actually gives people a false sense of security. Extra tweaks should be done to the Linux kernel. 

\subsection{Firewall Hardening}
Even though most Datacenters are already protected by hardware firewalls, software firewalls still have to be secure, to mitigate threats from the inside. The firewall should have a default inbound drop policy. By default only SSH should be allowed. 

\subsection{Banner Policy}
Though it might not seem as important, a custom banner hides sensitive system information and scares off some potential attackers. 

\subsection{Fail2Ban}
Fail2Ban is a great and easily customizable piece of software to reduce the amount of times someone can login and ban them after a certain amount of tries (Forever or for a given amount of time). Configuration happens in a jail.local file. 

\subsection{Password Policy}

Password expiry is outdated. Many big security advisories is advising against it: ...In this day and age, changing passwords every 90 days gives you the ILLUSION of stronger security while inflicting needless pain and cost to your organization... (SANS.org)

After asking a question to a board of infosec experts, they were able to give me the following answer:
If your passwords are strong (randomly generated by password managers) and one still manged to leak and find its way into the hands of an attacker, the solution is to fix the source of the leak, not to blindly update the passwords. Changing the passwords will not prevent new passwords to leak in the same way. It would be wiser to prevent password guessing attempts and to try detecting abnormal authentications. Using proper 2FA could also better mitigate passwords leaks.


\subsection{Default Umask}
This script changes the default permission from Permissive to Moderate. As recommended by CIS. 

\subsection{Blacklisting}
Unused and unsafe or deprecated modules/services should be blacklisted. The blacklist.conf file blacklists all the modules/services as recommended by CIS.


\subsection{SELinux Bolean values}
A SysAdmin should \textbf{never} disable SELinux, even though some sysadmins hate it. 
SELinux implements Mandatory Access Control (MAC). Every process and system resource has a special security label called a SELinux context. A SELinux context is an identifier, which abstracts away the system-level details and focuses on the security properties of the entity. Not only does this provide a consistent way of referencing objects in the SELinux policy, but it also removes any ambiguity that can be found in other identification methods; for example, a file can have multiple valid path names on a system that makes use of bind mounts. 





\subsection{Security Through Obscurity}
“Relying upon security through obscurity is bad: it usually leads to fragile or insecure systems”.
Some of the configuration rules I have implemented are just obfuscation rather than hardening. Many Info Security Experts believe that this is not a valid way of securing things, which is true, but it does add another layer of security. An extra layer of obscurity might get rid of Script-kiddies or less experienced hackers. Even experienced hackers will have to make more noise to get more information, which makes them more vulnerable to detection (IDS). 

Obscurity might add some additional security, but you should not rely upon it, and it should not be your primary defence. You should be prepared that the obscurity might be pierced, and be confident that you have adequate defences to handle that case.
This is a very controversial subject in the Cybersecurity world. After extensively reading the points that both parties make, I believe obscurity to be nothing more or less than an extra layer of security. Which is nothing bad in itself. The difficulties that more obscurity brings to the table are not of that much importance in this organisation, since only a select group of people will have to deal with these difficulties. 


\subsection{Unaccepted Best Practices}
This is a list of best practices I chose not to implement and why. 
\subsubsection{Disabling the root user}
I decided to make sure that people cannot login as the root user, but can still change to the root user when inside the system. Since experienced SysAdmins alone will use the system’s cli, I believe that we do not have to disable the root user as a whole. 

\subsubsection{Automatic  Security Updates}
After talking to my supervisors, we decided not to include any automatic patches/updates. This does increase the risk of a system becoming out of date or vulnerable. On the other hand, there is no risk that a production environment will crash because of a patch/update. I decided not to include the configuration files for this. Spacewalk will handle these upgrades (for now) so the systems will be kept up to date. We should look into automation for this. 