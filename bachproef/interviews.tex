
\section{Orientatie}

Momenteel wordt er naar 4 systemen gekeken voor dit onderzoek:
\begin{itemize}
	\item Safe Exam Browser 
	\item Televic AssessmentQ / AVIDAnet Lite
	\item Desktop virtualisatie via een cloudprovider
	\item Beveiligde, configureerbare netwerkomgeving om internettoegang tot niet-toegestane sites vanop laptops te vermijden
\end{itemize}
De vragen die gesteld worden tijdens dit interview zijn bedoeld om af te toetsen welk systeem het meest geschikt zou blijken volgens docenten om het huidige "systeem" met o.a. Netop School te vervangen. 
\newpage
\section{Vragen}

\begin{itemize}
	\item Neemt u computerexamens af op Hogechool Gent? Zoja, welke vakken? 
	\bigskip 
	\item Wat voor soort computerexamen neemt u af?
	\begin{itemize}
		\item Een online test waarbij antwoorden via de webbrowser invgevuld worden (bv. via Chamilo, Cisco-platform, enz.)
		\item Een schriftelijk examen waaarbij de voorbereiding op pc gebeurt maar de antwoorden op papier ingevuld worden.
		\item Een examen dat op computer m.b.v. specifieke software (bv. IDE en compiler) gemaakt wordt waarna het resultaat digitaal ingediend wordt (bv. Word document met antwoorden, zip-bestand met broncode, Github, ...)
		\item  Andere (te specifiëren)
		
	\end{itemize}
	\bigskip 

	\item Ik overloop nu enkele functies waarover een byod examensysteem zou kunnen beschikken. Kan u aangeven of deze essentieel (Must have), belangrijk, maar niet essentieel (Should have), nuttig, maar niet belangrijk (Could have) of onbelangrijk (Won't have) zijn?
	\begin{itemize}
		\item Toegang tot het internet kan verboden worden
		\item Beperken van toegang tot documenten
		\item Beperken van toegang tot software
		\item Digitale communicatie met medestudenten is niet mogelijk
		\item Monitoring tijdens het examen
		\item Snel en makkelijk opzetten van examens. 
		
\end{itemize}	
	\bigskip 
	\item Welke functies ontbreken momenteel de huidige manier waarop examens, op de computers van HoGent, afgenomen worden?
	\bigskip 
	\item Hoelang duurt het momenteel om een examen op computer voor te bereiden en hoe wordt dit gedaan (Computerlokaal voorbereiden, juiste examenfiles op de computer plaatsen) en hoe lang duurt het achteraf om alle examens op te halen?
	\bigskip 
	\item De huidige manier om examens af te leggen verloopt niet altijd zonder problemen. Kan u enkele defecten opsommen die u reeds tegengekomen bent? Moet u soms extra werk doen om alle examens op te halen (alle of enkele computers manueel afgaan)? Loopt er soms iets mis 10 minuten voor examen (softwarepaketten die ontbreken, examens die verdwenen zijn)? 
	\bigskip
	\item (Specifiek voor de docenten die programmeerexamens afleggern) Ziet u iets in het systeem van GitHub Classroom? Daarmee is het mogelijk om studenten gecontroleerd toegang te geven tot een opgave, een deadline op te leggen voor het indienen hiervan en zelfs automatisch te testen (m.b.v. een DevOps pipeline) en graden. Vergeleken met het huidige systeem (Word document) zou dit een veel meer efficiënte en veilige manier zijn.
	\bigskip 
	\item Heeft u nog vragen, Ideeën of opmerkingen over het nieuwe systeem?
	
\end{itemize}

\section{Samenvatting van het Interview met Lector Heidi Roobrouck}

Mevrouw Roobrouck legt met enkele collega's de examens van OOPI (Object-georiënteerd programmeren 1), OOPII (Object-georiënteerd programmeren 2) en OOPIII (Object-georiënteerd programmeren 3) af. Deze drie examens worden op dezelfde manier afgelegd, namelijk: volledig op computer, met behulp van een IDE (Netbeans), de Java API en een handboek programmeren. Zo een examen wordt ingediend door op het einde een Microsoft Word document in te dienen waarin de student de gevraagde stukken code geplakt heeft. Deze documenten worden verzameld na het examen om daarna door de lectoren gecorrigeerd te worden. 
\medskip 

Deze manier van werken omvat veel manueel werk voor de lectoren. Volgens mevrouw Roobrouck komt er nog een uur tot anderhalf uur voorbereiding aan te pas de dag van het exam. Om er zeker van te zijn dat alle nodige (examen)documenten op de computer staan en dat alle studenten direct an hun examen kunnen beginnen wanneer ze binnenkomen. Na het examen moeten de lectoren zich spoeden om op tijd weg te zijn, ofwel omdat ze zelf nog een examen geven die dag, of omdat het lokaal nog door een andere lector gebruikt gaat worden om een examen af te nemen. Volgens mevrouw Roobrouck kan het proces na het examen tot zo'n drie uur duren.  

Momenteel nemen de meeste lectoren de verantwoordelijkheid op hun om alle examens op te halen. Het huidige systeem kan enkel zip files op de juiste plaats met de juiste naam ophalen. Niet alle studenten slagen er in om de instructies te volgen en op een klas van 40 studenten zijn er telkens 2 à 3 examens die niet automatisch opgehaald kunnen worden. De docenten overlopen alle computers manueel om er zeker van te zijn dat iedereen die een examen gemaakt heeft ook effectief ingediend heeft, indien de student een bestand een verkeerde naam gegeven heeft zorgen de lectoren ervoor dat hij toch kan indienen. Het is natuurlijk niet logisch dat de verantwoordelijkheid bij het indienen van examens bij de lectoren ligt.

Na een examen moet lectoren ook nog alle files manueel van de computers weghalen, om er zeker van te zijn dat een student tijdens een volgend examen geen toegang heeft tot reeds gemaakt examens/opgaven. Het is duidelijk dat er een te grote overhead is, en het huidige systeem niet houdbaar is.

Mevrouw Roobrouck is geïnteresseerd in het fenomeen van GitHub Classroom maar zal eerst een goede werkende opstelling moeten zien om overtuigd te zijn van het concept. Het zou wel de verantwoordelijkheid voor het indienen van examens terug op de juiste plaats, bij de studenten, plaatsen. \newpage

\section{Samenvatting van het interview met systeembeheerder Chris Arents}

Meneer Arents deelde zijn antwoord op in 2 delen. 
\subsection{Systeembeheer}

Naar beheer toe moet er altijd een andere package  geschreven worden wanneer er een nieuwe versie uitkomt. Een nadeel hierbij is dat je organisatie al iemand moet hebben die dit onder de knie heeft en dat er niet altijd voldoende tijd geïnvesteerd wordt in het vooraf uittesten van de package.

Een nadeel van het huidige netsupport systeem is het niet kunnen uitsluiten van fraude. Je kan veel fraude vermijden maar niet echt alles. Internet kan uitgezet worden maar de netwerkschijven van de studenten kunnen nog steeds gekoppeld worden. Gelukkig weten de studenten dit niet.

\subsection{Lectoren}

Er moeten enkel scripts geschreven worden voor het uitpakken van je materiaal. Je kan de examens hiermee ook terug ophalen op de lectoren pc of op een server. Hiervoor dienen dan weer mappen gemaakt te worden met rechten. Vroeger werd dit door de systeembeheerders gedaan. Momenteel is dit de taak van de docent geworden.